\documentclass[10pt,letterpaper]{LabHMO}
\usepackage[spanish,activeacute,es-tabla]{babel}
\usepackage[utf8]{inputenc}
\usepackage{latexsym,amsfonts,amsmath,mathrsfs,amssymb}
\usepackage{fancyhdr}
\usepackage{multicol}        % used for the two-column index
\usepackage{multirow}        % used for the two-column index
\usepackage{fancybox}% http://ctan.org/pkg/fancybox
\usepackage{lastpage}
\usepackage{setspace}
\usepackage[usenames,dvipsnames]{xcolor}
\usepackage{pifont}
\usepackage{fixltx2e}
\usepackage{hhline}
\usepackage{colortbl}
\usepackage[final]{pdfpages}
\usepackage[pass]{geometry}
\usepackage{setspace}
\usepackage{picinsHtec}% http://ctan.org/pkg/picins
\usepackage[font=small, labelsep=period, justification=centering]{caption}% http://ctan.org/pkg/caption
\usepackage[pdftex, plainpages = false, pdfpagelabels,
                 pdfpagelayout = useoutlines,
                 bookmarks,
                 bookmarksopen = true,
                 bookmarksnumbered = true,
                 breaklinks = true,
                 linktocpage,
                 pagebackref,
                 colorlinks = false,  % was true
                 linkcolor = blue,
                 urlcolor  = blue,
                 citecolor = red,
                 anchorcolor = green,
                 hyperindex = true,
                 hyperfigures,
                 pdftitle={Actividad},
                 pdfauthor={Diego Espíndola - diegoespindola24@aragon.unam.mx},
                 pdfsubject={Universidad Nacional Aut\'onoma de M\'exico},
                 pdfstartview={Fit},
                 pdfkeywords={Rob\'otica}
                 ]{hyperref}
\usepackage{graphicx}
\usepackage{enumitem}
\usepackage{wrapfig}
\usepackage{blindtext}
\usepackage{algorithmic}
\usepackage{forloop}
\usepackage[pdftex]{eforms}
\usepackage{pgf}
\usepackage{subcaption}
\usepackage{wrapfig}
\usepackage{listings}
\definecolor{codegreen}{rgb}{0,0.6,0}
\definecolor{codegray}{rgb}{0.5,0.5,0.5}
\definecolor{codepurple}{rgb}{0.58,0,0.82}
\definecolor{backcolour}{rgb}{0.98,0.98,0.98}
\renewcommand{\lstlistingname}{Código}
\lstdefinestyle{DelphiHtec}{
    backgroundcolor=\color{backcolour},
    commentstyle=\color{codegreen},
    keywordstyle=\color{blue},
    numberstyle=\tiny\color{codegray},
    stringstyle=\color{codepurple},
    basicstyle=\linespread{0.9}\ttfamily\scriptsize,
    breakatwhitespace=false,
    breaklines=true,
    captionpos=b,
    keepspaces=true,
    numbers=left,
    numbersep=8pt,
    showspaces=false,
    showstringspaces=false,
    showtabs=false,
    tabsize=5,
    frame=trBL
}
\lstset{basewidth=0.5em,frameround=fttt, style=DelphiHtec}
\setlist[enumerate]{leftmargin=*}
\renewcommand \thesection{\Roman{section}}
\usepackage{backrefx}
\renewcommand*{\backrefpagesname}{\scriptsize {Citado en página}~}
\renewcommand{\backrefpagesnames}{\scriptsize {Citado en páginas}~}
\renewcommand{\backreflist}{\scriptsize {\space y}~}
\bibliographystyle{../ClasesGlobal/apalike_Lab_Htec}
%Hoovertec 2017
\setlength{\oddsidemargin}{-1.14cm}%-0.4 0.99 -1.22cm
\setlength{\evensidemargin}{-1.14cm}%-0.4 0.99 -1.22cm
\setlength{\topmargin}{-1.757cm}%-2.31 3.058 1.757
\setlength{\headheight}{2.515cm}%2.515
\setlength{\headsep}{-1.0mm}
\setlength{\textheight}{23.6cm}%23.5
\setlength{\textwidth}{18.8cm}%17.73 18.45

\newenvironment{changemargin}[2]{%
\begin{list}{}{%
\setlength{\leftmargin}{#1}%
\setlength{\rightmargin}{#2}%
}
\item[]}{\end{list}}

\setcounter{page}{0}

\newcommand{\placetextbox}[3]{% \placetextbox{<horizontal pos>}{<vertical pos>}{<stuff>}
  \setbox0=\hbox{#3}% Put <stuff> in a box
  \AddToShipoutPictureFG*{% Add <stuff> to current page foreground
    \put(\LenToUnit{#1\paperwidth},\LenToUnit{#2\paperheight}){\vtop{{\null}\makebox[0pt][c]{#3}}}%
  }%
}%

\everyTextField{\textFont{Times}
%    \AA{\AAFormat{console.println("Font set to: '" + event.target.textFont + "', trying to set it to: '" + event.target.value + "'"); event.target.textFont = event.target.value;}}
%\TU{J\"{u}rgen, press me and see what happens!}
    \Ff{\FfDoNotScroll}
}
%\usepackage{enumitem}
\graphicspath{{Imagenes/}} 
\begin{document}
	\newgeometry{textwidth=20.5cm}
\pagestyle{fancy}
\count1=113
\count3=115
\count2=1
\count4=1
\multiply \count2 by 9
\advance \count1 by -\count2
\multiply \count4 by 9
\advance \count3 by -\count4
\renewcommand{\headrulewidth}{0pt}
\fancyput(8.29cm,-11.436cm){%
\setlength{\unitlength}{1in}\hspace{-0.03cm}\fancyoval(7.7,10.4)\hspace{9.78cm}\rule[\count1 mm]{14mm}{7.1mm}%
\hspace{-1.397cm}\color{gray}{\rule[10.4cm]{0.5mm}{7.1mm}}%
\hspace{-0.05cm}\color{gray}{\rule[9.5cm]{0.5mm}{7.1mm}}%
\hspace{-0.05cm}\color{gray}{\rule[8.6cm]{0.5mm}{7.1mm}}%
\hspace{-0.05cm}\color{gray}{\rule[7.7cm]{0.5mm}{7.1mm}}%
\hspace{-0.05cm}\color{gray}{\rule[6.8cm]{0.5mm}{7.1mm}}%
\hspace{-0.05cm}\color{gray}{\rule[5.9cm]{0.5mm}{7.1mm}}%
\hspace{-0.05cm}\color{gray}{\rule[5.0cm]{0.5mm}{7.1mm}}%
\hspace{-0.05cm}\color{gray}{\rule[4.1cm]{0.5mm}{7.1mm}}%
\hspace{-0.05cm}\color{gray}{\rule[3.2cm]{0.5mm}{7.1mm}}%
\hspace{-0.05cm}\color{gray}{\rule[2.3cm]{0.5mm}{7.1mm}}%
\hspace{-0.05cm}\color{gray}{\rule[1.4cm]{0.5mm}{7.1mm}}%
\hspace{-0.05cm}\color{gray}{\rule[0.5cm]{0.5mm}{7.1mm}}%
\hspace{-0.05cm}\color{gray}{\rule[-0.4cm]{0.5mm}{7.1mm}}%
\hspace{-0.05cm}\color{gray}{\rule[-1.3cm]{0.5mm}{7.1mm}}%
\hspace{-0.05cm}\color{gray}{\rule[-2.2cm]{0.5mm}{7.1mm}}%
\hspace{-0.05cm}\color{gray}{\rule[-3.1cm]{0.5mm}{7.1mm}}%
\hspace{-0.05cm}\color{gray}{\rule[-4.0cm]{0.5mm}{7.1mm}}%
\hspace{-0.05cm}\color{gray}{\rule[-4.9cm]{0.5mm}{7.1mm}}%
\hspace{-0.05cm}\color{gray}{\rule[-5.8cm]{0.5mm}{7.1mm}}%
\hspace{-0.05cm}\color{gray}{\rule[-6.7cm]{0.5mm}{7.1mm}}%
\hspace{-0.05cm}\color{gray}{\rule[-7.6cm]{0.5mm}{7.1mm}}%
\hspace{-0.05cm}\color{gray}{\rule[-8.5cm]{0.5mm}{7.1mm}}%
\hspace{-0.05cm}\color{gray}{\rule[-9.4cm]{0.5mm}{7.1mm}}%
\hspace{-0.05cm}\color{gray}{\rule[-10.3cm]{0.5mm}{7.1mm}}%
\hspace{-0.05cm}\color{gray}{\rule[-11.2cm]{0.5mm}{7.1mm}}%
\hspace{-0.06cm}\color{black}{\rule[11.592cm]{13.95mm}{7.55mm}}%
\hspace{-1.385cm}\color{gray}{\rule[11.592cm]{0.5mm}{7.55mm}}%
\hspace{0.6mm}\raisebox{118.3 mm}{\color{white}{\textbf{\textsf{\begin{large}\makebox[8.5mm][c]{\hspace{1.2mm}\textField[\Ff{\FfReadOnly}\textColor{1 1 1}\Q{1}\BG{0 0 0}\textFont{Helvetica-Bold}\textSize{11.5}]{Iniciales}{0.85cm}{0.6cm}
}\end{large}}}}}
\hspace{-9.675mm}\raisebox{\count3 mm}{\color{white}{\textbf{\textsf{\begin{large}\makebox[8.5mm][c]{JDC}\end{large}}}}}
\hspace{-21.6941cm}\color{black}\rule[\count1 mm]{10.5mm}{7.1mm}%
\hspace{-0.05cm}\color{gray}{\rule[10.4cm]{0.5mm}{7.1mm}}%
\hspace{-0.05cm}\color{gray}{\rule[9.5cm]{0.5mm}{7.1mm}}%
\hspace{-0.05cm}\color{gray}{\rule[8.6cm]{0.5mm}{7.1mm}}%
\hspace{-0.05cm}\color{gray}{\rule[7.7cm]{0.5mm}{7.1mm}}%
\hspace{-0.05cm}\color{gray}{\rule[6.8cm]{0.5mm}{7.1mm}}%
\hspace{-0.05cm}\color{gray}{\rule[5.9cm]{0.5mm}{7.1mm}}%
\hspace{-0.05cm}\color{gray}{\rule[5.0cm]{0.5mm}{7.1mm}}%
\hspace{-0.05cm}\color{gray}{\rule[4.1cm]{0.5mm}{7.1mm}}%
\hspace{-0.05cm}\color{gray}{\rule[3.2cm]{0.5mm}{7.1mm}}%
\hspace{-0.05cm}\color{gray}{\rule[2.3cm]{0.5mm}{7.1mm}}%
\hspace{-0.05cm}\color{gray}{\rule[1.4cm]{0.5mm}{7.1mm}}%
\hspace{-0.05cm}\color{gray}{\rule[0.5cm]{0.5mm}{7.1mm}}%
\hspace{-0.05cm}\color{gray}{\rule[-0.4cm]{0.5mm}{7.1mm}}%
\hspace{-0.05cm}\color{gray}{\rule[-1.3cm]{0.5mm}{7.1mm}}%
\hspace{-0.05cm}\color{gray}{\rule[-2.2cm]{0.5mm}{7.1mm}}%
\hspace{-0.05cm}\color{gray}{\rule[-3.1cm]{0.5mm}{7.1mm}}%
\hspace{-0.05cm}\color{gray}{\rule[-4.0cm]{0.5mm}{7.1mm}}%
\hspace{-0.05cm}\color{gray}{\rule[-4.9cm]{0.5mm}{7.1mm}}%
\hspace{-0.05cm}\color{gray}{\rule[-5.8cm]{0.5mm}{7.1mm}}%
\hspace{-0.05cm}\color{gray}{\rule[-6.7cm]{0.5mm}{7.1mm}}%
\hspace{-0.05cm}\color{gray}{\rule[-7.6cm]{0.5mm}{7.1mm}}%
\hspace{-0.05cm}\color{gray}{\rule[-8.5cm]{0.5mm}{7.1mm}}%
\hspace{-0.05cm}\color{gray}{\rule[-9.4cm]{0.5mm}{7.1mm}}%
\hspace{-0.05cm}\color{gray}{\rule[-10.3cm]{0.5mm}{7.1mm}}%
\hspace{-0.05cm}\color{gray}{\rule[-11.2cm]{0.5mm}{7.1mm}}%
\hspace{-1.05cm}\color{black}{\rule[11.592cm]{10.5mm}{7.55mm}}%
\hspace{-0.05cm}\color{gray}{\rule[11.592cm]{0.5mm}{7.55mm}}%
\hspace{-1.05cm}\raisebox{118.3 mm}{\color{white}{\textbf{\textsf{\begin{large}\makebox[10.5mm][c]{\hspace{1.2mm}\textField[\Ff{\FfReadOnly}\textColor{1 1 1}\Q{1}\BG{0 0 0}\textFont{Helvetica-Bold}\textSize{11.5}]{Iniciales}{0.85cm}{0.6cm}
}\end{large}}}}}
\hspace{-1.18cm}\raisebox{\count3 mm}{\color{white}{\textbf{\textsf{\begin{large}\makebox[10.5mm][c]{T}\end{large}}}}
}}
\fancyhead{}
\fancyfoot{}
\fancyhead[C]{
\setlength\tabcolsep{4pt}
\hspace{-1.7cm}\vspace{4mm}
\begin{tabular}{c|l c c |c |c|c c l l l}
\multirow{5}{17mm}
{\hspace{.5mm}\includegraphics[height=1.5 cm]{FES-Logo.png}} & \multicolumn{3}{c|}{\multirow{2}{12.685cm}{\vspace*{-1.5mm}\begin{spacing}{0.7}\begin{scriptsize}\hspace*{45mm}{\normalsize {\large Róbotica }}\end{scriptsize}\\[-0.7mm]\centering\textsc{}\end{spacing}}}& \multicolumn{2}{p{2.04cm}|}{\begin{scriptsize}{\normalsize Grupo: 2007}\end{scriptsize}} &\multicolumn{2}{p{2cm}}{\hspace{4mm}\begin{scriptsize}{\bf 2023-2} \end{scriptsize}}\\
\cline{5-10}
& & & &\multicolumn{5}{p{2.04cm}}{ {\fontsize{7}{6}\selectfont {\normalsize Fecha:07/03/23}}}\\
\cline{2-11}
\vspace*{-1mm} &\multicolumn{3}{c|}{\multirow{1}{12.685cm}{\vspace*{-0mm}\begin{spacing}{0.5}\begin{scriptsize}\hspace*{-0.9mm}{\normalsize Nombre: Fernández Barbosa Luis Antonio}\end{scriptsize}\end{spacing}}} &\multicolumn{6}{l}{\hspace{0mm}\begin{scriptsize}{\normalsize Puntos:}\end{scriptsize}}\\
& \multicolumn{3}{c|}{\multirow{2}{12.685cm}{\vspace*{-4.5mm}\begin{spacing}{0.85}\centering\textbf{\textsf{{}}}\end{spacing}}}& \multicolumn{6}{l}{\multirow{1.2}{*}{\hspace{-1mm}\makebox[4.24cm][l]{\begin{scriptsize}\textsf{{
}}\end{scriptsize}}}}\\
\cline{5-11}
& & & & \multicolumn{7}{p{42.2mm}}{\begin{scriptsize}\textsf{Profesor: Dr. Jos\'e Daniel Castro D\'iaz}\end{scriptsize}}\\
\hline \hline
\end{tabular}

%\begin{tabular}{l|c|c|c|c|p{4mm}|}
%\multirow{5}{25mm}{\centering\includegraphics[height=2.2cm]{../ClasesGlobal/FIUNAMLogoColor}} & \multicolumn{2}{c|}{\multirow{5}{8.89cm}{\centering{\textbf{\textsf{Manual de prácticas del Laboratorio de \\ \laMateria}}}}} & \multirow{1}{30mm}{\centering\textsf{Código:}} & \multicolumn{1}{c}{\multirow{1}{39.4mm}{\centering\textsf{\elCodigo}}} \\ \cline{4-5}
% & \multicolumn{2}{c|}{} & \textsf{Versión:} & \multicolumn{2}{c}{\centering{\textsf{\laVersion}}}  \\ \cline{4-6}
% & \multicolumn{2}{c|}{} & \textsf{Página:} & \multicolumn{2}{c}{\textsf{\thepage \hspace{0.1cm} de \pageref{FinGuia}}} \\ \cline{4-6}
% & \multicolumn{2}{c|}{} & \cellcolor{light-gray}\textsf{Sección ISO:} & \multicolumn{2}{c|}{\cellcolor{light-gray}\textsf{8.3}}\\ \cline{4-6}
% & \multicolumn{2}{c|}{} & \textsf{Fecha de emisión:} & \multicolumn{2}{c}{\multirow{1}{40mm}{\centering\textsf{\laFechaEmision}}} \\  \cline{1-6}
%\multicolumn{2}{c|}{\multirow{1}{9cm}{\centering{\textsf{Facultad de Ingeniería}}}}  & \multicolumn{4}{c}{\textsf{Área/Departamento: Laboratorio de Automatización}} \\ \hline
%%\multicolumn{5}{c}{\textsf{La impresión de este documento es una copia no controlada}}\\ \hline \hline
%\end{tabular}
}
\ifpdf
    \pdfinfo { /Author (Diego Espíndla diegoespindola24@aragon.unam.mx)}
\fi
\restoregeometry
\vspace{-8mm} 
%\begin{center}
%\fbox{{\Large Ejercicio de Clase No.\hspace{0.7cm}}}
%\end{center}
\begin{center}
	\fbox{{\Large Tarea No. 4}}
\end{center}


\noindent {\large
Sean las matrices de rotación:

\begin{equation}
	{}^0\boldsymbol{R}_1=\begin{bmatrix}
		\cos\theta_1&-\sin\theta_1&0\\
		\sin\theta_1&\cos\theta_1&0\\
		0&0&1
	\end{bmatrix}
\end{equation}
y 

\begin{equation}
	{}^1\boldsymbol{R}_2=\begin{bmatrix}
		\cos\theta_2&-\sin\theta_2&0\\
		\sin\theta_2&\cos\theta_2&0\\
		0&0&1
	\end{bmatrix}
\end{equation}



\noindent obtener la matriz ${}^0\boldsymbol{R}_2$ si 

\begin{equation}
{}^0\boldsymbol{R}_2={}^0\boldsymbol{R}_1{}^1\boldsymbol{R}_2.
\end{equation}
Para el resultado final, considerar las identidades de suma y resta de dos ángulos:
\begin{align*}
	\sin(\alpha\pm\beta)=&\sin\alpha\cos\beta\pm\cos\alpha\sin\beta\\
	\cos(\alpha\pm\beta)=&\cos\alpha\cos\beta\pm\sin\alpha\sin\beta\\
	\tan(\alpha\pm\beta)=&\frac{\tan\alpha\pm\tan\beta}{1\pm\tan\alpha\tan\beta}.
\end{align*}
	
Resultado:

\begin{equation}
{}^0\boldsymbol{R}_2={}^0\boldsymbol{R}_1{}^1\boldsymbol{R}_2=
\end{equation}

\begin{equation}
	{}^0\boldsymbol{R}_2=\begin{bmatrix}
		\cos\theta_1&-\sin\theta_1&0\\
		\sin\theta_1&\cos\theta_1&0\\
		0&0&1
	\end{bmatrix}
		\begin{bmatrix}
		\cos\theta_2&-\sin\theta_2&0\\
		\sin\theta_2&\cos\theta_2&0\\
		0&0&1
	\end{bmatrix}=
\end{equation}

\begin{equation}
	=\begin{bmatrix}
		[(\cos\theta_1)(\cos\theta_2)+(-\sin\theta_1)(\sin\theta_2)+0] & [(\cos\theta_1)(-\sin\theta_2)+(-\sin\theta_1)(\cos\theta_2)+0] & 0\\
		[(\sin\theta_1)(\cos\theta_2)+(\cos\theta_1)(\sin\theta_2)+0] & [(\sin\theta_1)(-\sin\theta_2)+(\cos\theta_1)(\cos\theta_2)+0] & 0\\
		0&0&1
	\end{bmatrix}=
\end{equation}

\begin{equation}
	=\begin{bmatrix}
		[\cos\theta_1\cos\theta_2-\sin\theta_1\sin\theta_2] & [-\cos\theta_1\sin\theta_2-\sin\theta_1\cos\theta_2] & 0\\
		[\sin\theta_1\cos\theta_2+\cos\theta_1\sin\theta_2] & [-\sin\theta_1\sin\theta_2+\cos\theta_1\cos\theta_2] & 0\\
		0&0&1
	\end{bmatrix}=
\end{equation}

\begin{equation}
	=\begin{bmatrix}
		[\cos\theta_1\cos\theta_2-\sin\theta_1\sin\theta_2] & -[\cos\theta_1\sin\theta_2+\sin\theta_1\cos\theta_2] & 0\\
		[\sin\theta_1\cos\theta_2+\cos\theta_1\sin\theta_2] & [\cos\theta_1\cos\theta_2-\sin\theta_1\sin\theta_2] & 0\\
		0&0&1
	\end{bmatrix}=
\end{equation}

\begin{equation}
	\fbox{$^0\boldsymbol{R}_2=\begin{bmatrix}
		\cos(\theta_1-\theta_2) & -\sin(\theta_1+\theta_2) & 0\\
		\sin(\theta_1+\theta_2) & \cos(\theta_1-\theta_2) & 0\\
		0&0&1\\
	\end{bmatrix}$}
\end{equation}

\centerline{\textbf{Matriz de rotación} (articulación 2 respecto a articulación 0)}
}
\end{document}